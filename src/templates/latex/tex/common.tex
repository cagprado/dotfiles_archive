% vim: nospell
% LOAD PACKAGES #############################################################
% to select features depending on engine
\usepackage{ifpdf,ifluatex}

% allow use of . in filenames
\usepackage{grffile}

% spacing
\usepackage{indentfirst,xspace}

% basic graphics packages
\usepackage{graphicx,xcolor}

% tables
\usepackage{booktabs}
\renewcommand{\arraystretch}{1.2}  % more space between rows

% remove extra space at borders of table (lines are then justified with text)
\let\oldtabular\tabular
\let\endoldtabular\endtabular
\renewenvironment{tabular}[1]{\begin{oldtabular}{@{}#1@{}}}{\end{oldtabular}}

% basic mathematics packages
\usepackage{amsmath,amssymb}

% fonts
\ifluatex
  \usepackage{fontspec}
  \defaultfontfeatures{Ligatures=TeX}
  % Font shapes:
  % UprightFont, BoldFont,
  % ItalicFont, BoldItalicFont,
  % SlantedFont, BoldSlantedFont,
  % SmallCapsFont
  \setmainfont{LinLibertine_}[
    Extension = .otf,
    UprightFont = *R,
    BoldFont = *RB,
    ItalicFont = *RI,
    BoldItalicFont = *RBI,
    Numbers = {Proportional,OldStyle},
    SmallCapsFeatures = { LetterSpace=5, Kerning=On },
  ]
  \setsansfont{LinBiolinum_}[
    Extension = .otf,
    UprightFont = *R,
    BoldFont = *RB,
    ItalicFont = *RI,
    BoldItalicFont = *RBO,
    Numbers = {Proportional,OldStyle},
    SmallCapsFeatures = { LetterSpace=7, Kerning=On },
  ]
  \setmonofont{iosevkatermslab}[
    UprightFont = *nerdfontcomplete,
    BoldFont = *boldnerdfontcomplete,
    ItalicFont = *italicnerdfontcomplete,
    BoldItalicFont = *bolditalicnerdfontcomplete,
    SlantedFont = *obliquenerdfontcomplete,
    BoldSlantedFont = *boldobliquenerdfontcomplete,
    Scale=MatchLowercase,
  ]
  \usepackage{unicode-math}
  \setmathfont{Asana-Math.otf}

  % lining figures
  \newcommand\lfstyle{\addfontfeature{Numbers=Lining}}
  \newcommand\textlf[1]{{\lfstyle #1}}
\else
  \usepackage[T1]{fontenc}
  \usepackage[utf8]{inputenc}
  \usepackage[p,osf]{newtxtext}  % old-style figures in text mode
  \usepackage{newtxmath}
\fi
\newcommand\Arabic[1]{\textlf{\arabic{#1}}}  % Arabic uses lining figures

% more mathematics packages
\usepackage{siunitx,commath,xfrac}

% ASYMPTOTE PREAMBLE ########################################################
\ifluatex
  \ifx\pdfpagewidth\undefined\let\pdfpagewidth\paperwidth\fi
  \ifx\pdfpageheight\undefined\let\pdfpageheight\paperheight\fi
\else
  \let\paperwidthsave\paperwidth\let\paperwidth\undefined
  \let\paperwidth\paperwidthsave
\fi
\newbox\ASYbox
\newdimen\ASYdimen
\def\ASYprefix{}
\long\def\ASYbase#1#2{\leavevmode\setbox\ASYbox=\hbox{#1}%\ASYdimen=\ht\ASYbox%
\setbox\ASYbox=\hbox{#2}\lower\ASYdimen\box\ASYbox}
\long\def\ASYaligned(#1,#2)(#3,#4)#5#6#7{\leavevmode%
\setbox\ASYbox=\hbox{#7}%
\setbox\ASYbox\hbox{\ASYdimen=\ht\ASYbox%
\advance\ASYdimen by\dp\ASYbox\kern#3\wd\ASYbox\raise#4\ASYdimen\box\ASYbox}%
\setbox\ASYbox=\hbox{#5\wd\ASYbox 0pt\dp\ASYbox 0pt\ht\ASYbox 0pt\box\ASYbox#6}%
\hbox to 0pt{\kern#1pt\raise#2pt\box\ASYbox\hss}}%
\ifpdf
  \long\def\ASYalignT(#1,#2)(#3,#4)#5#6{%
  \ASYaligned(#1,#2)(#3,#4){%
  \special{pdf:q #5 0 0 cm}%
  }{%
  \special{pdf:Q}%
  }{#6}}
  \long\def\ASYalign(#1,#2)(#3,#4)#5{\ASYaligned(#1,#2)(#3,#4){}{}{#5}}
  \def\ASYraw#1{#1}
\else
  \long\def\ASYalignT(#1,#2)(#3,#4)#5#6{%
  \ASYaligned(#1,#2)(#3,#4){%
  \special{ps:gsave currentpoint currentpoint translate [#5 0 0] concat neg exch neg exch translate}%
  }{%
  \special{ps:currentpoint grestore moveto}%
  }{#6}}
  \long\def\ASYalign(#1,#2)(#3,#4)#5{\ASYaligned(#1,#2)(#3,#4){}{}{#5}}
  \def\ASYraw#1{
  currentpoint currentpoint translate matrix currentmatrix
  100 12 div -100 12 div scale
  #1
  setmatrix neg exch neg exch translate}
\fi
